\chapter{Introdução}
\label{chap:intro}

A digitalização tem transformado a forma como profissionais autônomos se apresentam e interagem com seus clientes. No contexto atual, onde a presença online é fundamental para a captação de clientes, o desenvolvimento de um portfólio digital se torna uma estratégia essencial para destacar as habilidades e serviços oferecidos. Este relatório apresenta o processo de criação de um portfólio digital para um profissional autônomo na área de manutenção residencial, abrangendo serviços como refrigeração, hidráulica, elétrica, alvenaria e acabamento.

% Este pode ser um parágrafo citado por alguém \cite{Barabasi2003-1} e \cite{barabasi2003linked}.
% Para ajustar veja o comentário do capítulo \ref{chap:fundteor}.

% As orientações do robô \cite{aperea-1}.

% fakdfjlsdjfldsjfldsj
% dfkhfdskfhkdjh


% Segundo \citeonline{barabasi2003linked}, ...

% 
% \loremipsum dolor sit amet, consectetur adipiscing elit. Sed do eiusmod tempor incididunt ut labore et dolore magna aliqua. Ut enim ad minim veniam, quis nostrud exercitation ullamco laboris nisi ut aliquip ex ea commodo consequat. Duis aute irure dolor in reprehenderit in voluptate velit esse cillum dolore eu fugiat nulla pariatur. Excepteur sint occaecat cupidatat non proident, sunt in culpa qui officia deserunt mollit anim id est laborum.
%--------- NEW SECTION ----------------------
\section{Objetivos}
\label{sec:obj}

O presente relatório tem como objetivo detalhar o processo de desenvolvimento de um portfólio digital para um profissional autônomo na área de manutenção residencial. Em um mercado cada vez mais digitalizado, a presença online tornou-se crucial para a visibilidade e captação de clientes. Para profissionais que atuam com serviços como refrigeração, hidráulica, elétrica, alvenaria e acabamento, um portfólio online não apenas apresenta os serviços de forma organizada, mas também transmite profissionalismo, confiança e a qualidade dos trabalhos realizados.

Este projeto visa suprir a necessidade do cliente em estabelecer uma presença digital robusta, permitindo que potenciais clientes conheçam seus serviços, certificações, experiência e entrem em contato de forma facilitada. O escopo do projeto abrange desde o planejamento estratégico e a definição da identidade visual até a implementação técnica do site, com foco em usabilidade, design responsivo e integração de funcionalidades de contato eficientes. O sucesso deste projeto será medido pela capacidade do portfólio de atrair e converter visitantes em leads qualificados, fortalecendo a marca pessoal do profissional no ambiente digital.

A iniciativa também promove a aplicação prática dos conteúdos teóricos abordados em sala de aula, permitindo à equipe o contato com processos de levantamento de requisitos, modelagem de sistemas, planejamento estratégico e desenvolvimento orientado ao usuário.

\label{sec:obj}

\subsection{Objetivos Específicos}
\label{ssec:objesp}

Os objetivos específicos deste projeto são:
\begin{itemize}
      \item Desenvolver habilidades de gestão de projetos.
      \item O cliente estabelecer uma presença digital robusta;
      \item Atrair e converter visitantes em leads qualificados;
      \item Transmitir confiança e qualidade no trabalho do profissional;
  \end{itemize}

\subsubsection*{Objetivos específicos principais}
\label{sssec:obj-principais}

Os objetivos principais deste projeto são:
\begin{itemize}
      \item Formalizar os requisitos funcionais e não funcionais do portfólio digital;
      \item Produzir artefatos técnicos como diagramas UML e matriz QFD;
      \item Desenvolver um protótipo funcional navegável e responsivo;
      \item Integrar elementos de identidade visual e ferramentas de contato;
      \item Validar o portfólio com feedback positivo do cliente.
  \end{itemize}

%--------- NEW SECTION ----------------------
\section{Justificativa}
\label{sec:justi}

A justificativa para o desenvolvimento deste portfólio digital reside na crescente digitalização dos processos comerciais e na mudança no comportamento do consumidor, que cada vez mais busca serviços online. Para profissionais autônomos, especialmente aqueles que atuam em áreas como manutenção residencial, a presença digital não é apenas uma vantagem competitiva, mas uma necessidade para alcançar novos clientes e consolidar sua marca pessoal.

Além disso, a criação de um portfólio digital permite que o profissional apresente seus serviços de forma clara e organizada, destacando suas certificações, experiências e projetos anteriores. Isso não apenas aumenta a visibilidade do profissional, mas também constrói confiança com potenciais clientes, que podem avaliar a qualidade do trabalho antes de entrar em contato.

%--------- NEW SECTION ----------------------
\section{Organização do documento}
\label{section:organizacao}

Este documento apresenta $5$ capítulos e está estruturado da seguinte forma:

\begin{itemize}

  \item \textbf{Capítulo  \ref{chap:intro} - Introdução}: Apresenta os objetivos do relatório, a justificativa para o desenvolvimento do portfólio digital e a organização do documento;
  \item \textbf{Capítulo  \ref{chap:fundteor} - Fundamentação Teórica}: Aborda os conceitos teóricos que fundamentam o projeto, incluindo Gestão de Requisitos, Design Centrado no Usuário e modelo Ágil de Projetos;
  \item \textbf{Capítulo  \ref{chap:metod} - Materiais e Métodos}: Descreve a metodologia utilizada para o desenvolvimento do portfólio digital, incluindo as etapas de levantamento de requisitos, modelagem e prototipagem;
  \item \textbf{Capítulo  \ref{chap:result} - Resultados}: Apresenta os resultados obtidos com o desenvolvimento do portfólio digital, incluindo a validação com o cliente e as métricas de sucesso;
  \item \textbf{Capítulo  \ref{chap:conc} - Conclusão}: Apresenta as conclusões e contribuições.

\end{itemize}